% !Mode:: "TeX:UTF-8"
\documentclass[a4paper,12pt]{article}

%插入超链接,并且去除超链接的颜色下划线等特性
\usepackage[hidelinks]{hyperref}

%该宏包可以添加伪代码CLRS《算法导论》**第三版**模式
%clrscode3e.sty放在项目根目录下即可
\usepackage{clrscode3e}

%插入数学公式
\usepackage{amsmath}

%中文包
\usepackage{xeCJK}
\setCJKmainfont{SimSun}

%设置页边距
\usepackage[top=1in,bottom=0.8in,left=1.1in,right=1in]{geometry}

\usepackage{xcolor}

%为了插入代码
\usepackage{listings}

\definecolor{codegreen}{rgb}{0,0.6,0}
\definecolor{codegray}{rgb}{0.5,0.5,0.5}
\definecolor{codepurple}{rgb}{0.58,0,0.82}
\definecolor{backcolour}{rgb}{0.95,0.95,0.92}

\lstdefinestyle{mystyle}{
    backgroundcolor=\color{backcolour},
    commentstyle=\color{codegreen},
    keywordstyle=\color{magenta},
    numberstyle=\tiny\color{codegray},
    stringstyle=\color{codepurple},
    basicstyle=\footnotesize,
    breakatwhitespace=false,
    breaklines=true,
    captionpos=b,
    keepspaces=true,
    numbers=left,
    numbersep=5pt,
    showspaces=false,
    showstringspaces=false,
    showtabs=false,
    tabsize=2
}

\lstset{style=mystyle}
%%%%%%%%%%%%%%%%%%%%%%%%%%%%%%%%%

%首段缩进
\usepackage{indentfirst}

%为了画图
\usepackage{tikz}
\usetikzlibrary{positioning}
\usetikzlibrary{backgrounds}

\title{Assignment 1\\Algorithm Design and Analysis}

\author{\href{http://bitjoy.net}{bitjoy.net}}
\date{October 7, 2015}

\begin{document}

\maketitle

I choose problem 1,3,4,7,8.

\section*{1\quad Divide and Conquer}
\subsection*{\textnormal{1.1\quad Algorithm in natural language}}


Assume the two databases are A and B, A(i) and B(i) are the $i^{th}$ smallest value each contains.


First we compare the median of A and B. Let $k=\lfloor n/2 \rfloor$, so A(k) and B(k) are the median of A and B. If A(k) < B(k) (as all values are distinct, no A(k) == B(k) case; the case A(k) > B(k) is the same if we exchange A and B), then the median of combined 2n values must be in A[k,n] or B[1,k]. Because A(k) is greater than the first $k-1$ elements in A, B(k) > A(k), so the last $k$ elements of B are also greater than the first $k-1$ elements of A. As a result, the median of 2n values couldn't lie in A[1,k-1], neither B[k+1,n].


So, take A' = A[k,n] as the new A, B' = B[1,k] as the new B, but we can't delete the databases, the $i^{th}$ smallest value in A' is the $(i + k)^{th}$ smallest value in A, the $i^{th}$ smallest value in B' is the $i^{th}$ smallest value in B, recursively we will get the median of combined 2n values.

\subsection*{\textnormal{1.2\quad Algorithm in pseudo-code}}

We define algorithm MEDIAN(n,a,b) that input integers n,a and b and output the median of the union of the two parts A[a+1,b+n] and B[b+1,b+n].

\begin{codebox}
\Procname{$\proc{MEDIAN}(n,a,b)$}
\li \If n == 1
    \Then
\li \Return min(A(a+k),B(b+k))
 	\End
\li $k = \lfloor n/2 \rfloor$
\li \If A(a + k) < B(b + k)
    \Then
\li \Return MEDIAN(k, a + k, b)
\li \Else
\li \Return MEDIAN(k, a, b + k)
	\End
\end{codebox}

To find the median of 2n elements in A and B, we just call MEDIAN(n,0,0).

\subsection*{\textnormal{1.3\quad Subproblem reduction graph}}

\begin{center}
\tikzstyle{subnode}=[circle,inner sep=0pt,minimum size=0.4cm,scale=1.5,draw=blue!50,fill=blue!20,thick]
\tikzstyle{subsubnode}=[circle,inner sep=0pt,minimum size=0.4cm,scale=1.5,draw=white,fill=white,thick]
\tikzstyle{minnode}=[circle,inner sep=0pt,minimum size=0.4cm,scale=1.5,draw=red!50,fill=red!90,thick]
\begin{tikzpicture}[auto]
\node at ( 0,7) [subnode,label=above:{$MEDIAN(n,0,0)$}] (first){};
\node at ( -2,5) [subnode,label=left:{$MEDIAN(\lfloor n/2 \rfloor,\lfloor n/2 \rfloor,0)$}] (second1){};
\node at ( 2,5) [subnode,label=right:{$MEDIAN(\lfloor n/2 \rfloor,0,\lfloor n/2 \rfloor)$}] (second2){};
\node at ( 1,3) [subsubnode] (third1){};
\node at ( 3,3) [subsubnode] (third2){};
\node at ( -1,3) [subsubnode] (third3){};
\node at ( -3,3) [subsubnode] (third4){};
\node at ( 0,0) [minnode,label=right:{$MEDIAN(1,i,j)=min(A(i),B(j))$}] (origin){};

\draw [->] (first) to node[swap]{$A[\lfloor n/2 \rfloor] < B[\lfloor n/2 \rfloor]$} (second1);
\draw [->] (first) to node{$A[\lfloor n/2 \rfloor] > B[\lfloor n/2 \rfloor]$} (second2);

\draw [->] (second1) to (third3);
\draw [->] (second1) to (third4);
\draw [->] (second2) to (third1);
\draw [->] (second2) to (third2);

\draw [->,dashed] (third1) to (origin);
\draw [->,dashed] (third2) to (origin);
\draw [->,dashed] (third3) to (origin);
\draw [->,dashed] (third4) to (origin);

\begin{pgfonlayer}{background}
\filldraw [line width=4mm,join=round,black!10]
(8,-0.5)rectangle (-8,8);
\end{pgfonlayer}

\end{tikzpicture}
\end{center}

\subsection*{\textnormal{1.4\quad Correctness of the algorithm}}

We can use \textbf{\textit{loop invariant}} to prove it.

\textbf{Initialization:} At the beginning, we call MEDIAN(n,0,0) to find the median of the union of A[1,n] and B[1,n], say it's $M_1$. As described in the section 1.1, we know $M_2=MEDIAN(\lfloor n/2 \rfloor,\lfloor n/2 \rfloor,0)$, the median of the union of $A[\lfloor n/2 \rfloor,n]$ and $B[1,\lfloor n/2 \rfloor]$, is also the median of the union of A[1,n] and B[1,n], that's to say $M_2==M_1$.

\textbf{Maintenance:} We take $A'=A[\lfloor n/2 \rfloor,n]$ as the new A, $B'=B[1,\lfloor n/2 \rfloor]$ as the new B, after calling $M_3=median(\lfloor n/4 \rfloor,\lfloor n/4 \rfloor,0)$, we can say $M_3==M_2$, so $M_3==M_1$.

\textbf{Termination:} After calling MEDIAN m(m is big enough) times, only A(i) and B(j) left, the median of them is $M_m=min(A(i),B(j))$, as $M_m==M_{m-1}==...==M_1$, $M_m$ is the final global median.


\subsection*{\textnormal{1.5\quad Complexity of the algorithm}}

Let T(n) be the number of queries asked by our algorithm, each time we call the function, we ask 2 queries in line 4, after that, half elements are "eliminated", so we have $T(n)=T(\lfloor n/2 \rfloor) + 2$. Therefore $T(n)=2\lfloor logn \rfloor=O(logn)$.

\section*{3\quad Divide and Conquer}
\subsection*{\textnormal{3.1\quad Algorithm in natural language}}

In the textbook, we can find the inversions while merging two sub-sorted array, because their conditions are the same. When $i<j$ and $a_i>a_j$,  $a_j$ should be in front of $a_i$ and $(a_i,a_j)$ is a inversion, so we do two things in one merge. But when counting \textit{significant inversions}, we can't do these in the same time, because $a_i>a_j$ doesn't mean $a_i>3a_j$.

Don't worry, we can do it in two merges, one for sorting, one for finding
\textit{significant inversions}.The new algorithm is very similar to the one in textbook, so let's go straight to section 3.2 to see the pseudo-code.
\subsection*{\textnormal{3.2\quad Algorithm in pseudo-code}}

We define algorithm SORT-AND-COUNT(A) that input an unsorted array A and output the number of \textit{significant inversions} in original A and the sorted array A.

\begin{codebox}
\Procname{$\proc{MERGE-AND-COUNT}(L,R)$}
\li RC = 0; i = 0; j = 0;
\li \For k = 0 \To $||L||+||R||-1$
    \Do
\li \If L[i] > R[j]
    \Then
\li A[k] = R[j]
\li j++
\li \Else
\li A[k] = L[i]
\li i++
	\End
	\End
\li i = 0; j = 0;
\li \For k = 0 \To $||L||+||R||-1$
    \Do
\li \If L[i] > 3R[j]
    \Then
\li RC = RC + \id{length}[L] - i
\li j++
\li \Else
\li i++
	\End
    \End
\li \Return (RC,A)
\end{codebox}

\begin{codebox}
\Procname{$\proc{SORT-AND-COUNT}(A)$}
\li \If A has one element
    \Then
\li \Return (0,A)
\li \Else
\li Divide A into two sub-sequences L and R
\li ($RC_L$,L) = SORT-AND-COUNT(L)
\li ($RC_R$,R) = SORT-AND-COUNT(R)
\li (C,A) = MERGE-AND-COUNT(L,R)
\li \Return ($RC=RC_L+RC_R+C$,A)
\End
\end{codebox}

\subsection*{\textnormal{3.3\quad Subproblem reduction graph}}

SAC(A) is short for SORT-AND-COUNT(A).

\begin{center}
\tikzstyle{subnode}=[circle,inner sep=0pt,minimum size=0.4cm,scale=1.5,draw=blue!50,fill=blue!20,thick]
\tikzstyle{subsubnode}=[circle,inner sep=0pt,minimum size=0.4cm,scale=1.5,draw=white,fill=white,thick]
\tikzstyle{minnode}=[circle,inner sep=0pt,minimum size=0.4cm,scale=1.5,draw=red!50,fill=red!90,thick]
\begin{tikzpicture}[auto]
\node at ( 0,7) [subnode,label=above:{$SAC(A[1,n])$}] (first){};
\node at ( -2,5) [subnode,label=left:{$SAC(A[1,\lfloor n/2 \rfloor])$}] (second1){};
\node at ( 2,5) [subnode,label=right:{$SAC(A[\lceil n/2 \rceil,n])$}] (second2){};
\node at ( 1,3) [subsubnode] (third1){};
\node at ( 3,3) [subsubnode] (third2){};
\node at ( -1,3) [subsubnode] (third3){};
\node at ( -3,3) [subsubnode] (third4){};
\node at ( 1,0) [minnode] (leaf1){};
\node at ( 3,0) [minnode] (leaf2){};
\node at ( -1,0) [minnode,label=below:{$SAC(A[i])~return~(0,A[i])$}] (leaf3){\tiny A[i]};
\node at ( -3,0) [minnode,label=left:{$leaves:\qquad$}] (leaf4){};

\draw [->] (first) to (second1);
\draw [->] (first) to (second2);

\draw [->] (second1) to (third3);
\draw [->] (second1) to (third4);
\draw [->] (second2) to (third1);
\draw [->] (second2) to (third2);

\draw [->,dashed] (third1) to (leaf1);
\draw [->,dashed] (third2) to (leaf2);
\draw [->,dashed] (third3) to (leaf3);
\draw [->,dashed] (third4) to (leaf4);

\begin{pgfonlayer}{background}
\filldraw [line width=4mm,join=round,black!10]
(8,-1)rectangle (-8,8);
\end{pgfonlayer}

\end{tikzpicture}
\end{center}

\subsection*{\textnormal{3.4\quad Correctness of the algorithm}}
As we can see in section 3.2, the new MERGE-AND-COUNT just add an extra merge based on the old MERGE-AND-COUNT in the textbook, so it is obvious that new algorithm can sort array correctly.

As for finding all \textit{significant inversions}, suppose we are going to merge $L[1,n_1]$ and $R[1,n_1]$ which are already sorted. If $L[i] > 3R[j]$, then $(L[i],R[j])$ is a \textit{significant inversion}, as all $L[i+1,n_1]$ is greater than $L[i]$, so $L[i+1,n_1]$ together with $R[j]$ are \textit{significant inversions} too. Thus, the number of \textit{significant inversions} is $n_1-i$.

So, recursively we can find all \textit{significant inversions}.

\subsection*{\textnormal{3.5\quad Complexity of the algorithm}}

Let T(n) be the time of my algorithm, as there are an extra merge in the new MERGE-AND-COUNT algorithm, so the MERGE-AND-COUNT time is O(2n), we get $T(n)=2T(n/2)+O(2n)$, thus $T(n)=O(nlgn)$.

\section*{4\quad Divide and Conquer}
\subsection*{\textnormal{4.1\quad Algorithm in natural language}}
Given the complete binary tree T, let t be the root of T, $t_L$ and $t_R$ be the left and right child of t.

If $t<t_L$ and $t<t_R$, t is one of \textit{local minimum} node; if not, choose one of child that less than t, say $t_L$(or $t_R$), check if $t_L$'s 2 children are less than t, if so, $t_L$ is the \textit{local minimum} node; if not, recursively check one of $t_L$'s child.

If we can't find a \textit{local minimum} node among T's internal nodes, assume we reach X which is greater than its left child $X_L$, and $X_L$ is a leaf node, as a result, $X_L$ is the \textit{local minimum} node.

So, we can always find a \textit{local minimum} node.

\subsection*{\textnormal{4.2\quad Algorithm in pseudo-code}}

We define algorithm FIND-LOCAL-MINIMUM(X) that input a node X and output one of \textit{local minimum} node in X's subtree.

\begin{codebox}
\Procname{$\proc{FIND-LOCAL-MINIMUM}(X)$}
\li \If X has children
    \Then
\li Let $X_L$ and $X_R$ be the left and right child of X
\li \If $X<X_L$ and $X<X_R$
    \Then
\li \Return X
\li \ElseIf $X>X_L$
    \Then
\li \Return FIND-LOCAL-MINIMUM($X_L$)
\li \ElseIf $X>X_R$
    \Then
\li \Return FIND-LOCAL-MINIMUM($X_R$)
	\End
\li \Else
\li \Return X
	\End
\end{codebox}

To find the \textit{local minimum} node of T, we just call FIND-LOCAL-MINIMUN(T.root).
\subsection*{\textnormal{4.3\quad Subproblem reduction graph}}
FLM(X) is short for FIND-LOCAL-MINIMUM(X).

\begin{center}
\tikzstyle{subnode}=[circle,inner sep=0pt,minimum size=0.4cm,scale=1.5,draw=blue!50,fill=blue!20,thick]
\tikzstyle{subsubnode}=[circle,inner sep=0pt,minimum size=0.4cm,scale=1.5,draw=white,fill=white,thick]
\tikzstyle{minnode}=[circle,inner sep=0pt,minimum size=0.4cm,scale=1.5,draw=red!50,fill=red!90,thick]
\begin{tikzpicture}[auto]
\node at ( 0,7) [subnode,label=above:{$FLM(T.root)$}] (first){};
\node at ( -2,5) [subnode,label=left:{$FLM(T.root.left$)}] (second1){};
\node at ( 2,5) [subnode,label=right:{$FLM(T.root.right)$}] (second2){};
\node at ( 1,3) [subsubnode] (third1){};
\node at ( 3,3) [subsubnode] (third2){};
\node at ( -1,3) [subsubnode] (third3){};
\node at ( -3,3) [subsubnode] (third4){};
\node at ( 1,0) [minnode] (leaf1){};
\node at ( 3,0) [minnode] (leaf2){};
\node at ( -1,0) [minnode,label=below:{$FLM(x)~return~x$}] (leaf3){\small x};
\node at ( -3,0) [minnode,label=left:{$leaves:\qquad$}] (leaf4){};

\draw [->] (first) to node[swap]{$T.root>T.root.left$} (second1);
\draw [->] (first) to node{$T.root>T.root.right$} (second2);

\draw [->] (second1) to (third3);
\draw [->] (second1) to (third4);
\draw [->] (second2) to (third1);
\draw [->] (second2) to (third2);

\draw [->,dashed] (third1) to (leaf1);
\draw [->,dashed] (third2) to (leaf2);
\draw [->,dashed] (third3) to (leaf3);
\draw [->,dashed] (third4) to (leaf4);

\draw[line width=5pt,color=gray] (first) -- (second1) -- (third3) -- (leaf3);

\begin{pgfonlayer}{background}
\filldraw [line width=4mm,join=round,black!10]
(8,-1)rectangle (-8,8);
\end{pgfonlayer}

\end{tikzpicture}
\end{center}

\subsection*{\textnormal{4.4\quad Correctness of the algorithm}}

We can use \textbf{\textit{loop invariant}} to prove it.

\textbf{Initialization:} At the beginning, if $T.root<T.root.left$ and $T.root<T.root.right$, T itself is a \textit{local minimum} node.

\textbf{Maintenance:} Otherwise, claim that at any point in the execution of the algorithm, the parent (if any) of X has a greater value than X itself. Thus, X only need to compare with $X_L$ and $X_R$, if $X<X_L$ and $X<X_R$, X is a \textit{local minimum} node, if not, go to line 6 or line 8 recursively.

\textbf{Termination:} If algorithm doesn't return among internal nodes, it reaches leaf node X, so the parent of X has a greater value than X, thus, X is a \textit{local minimum} node.

So, the algorithm can always find a \textit{local minimum} node.

\subsection*{\textnormal{4.5\quad Complexity of the algorithm}}

As we can see in the section 4.3, each time we go to one of X's subtree and do 3 probes, as the longest path is the height of the tree, say $log_2n$. so we do at most $3log_2n$ probes, so the complexity is $O(logn)$.

\section*{7\quad Divide and Conquer}
\subsection*{\textnormal{7.1\quad Implementation of the Sort-and-Count algorithm}}

I implemented the Sort-and-Count algorithm in Python3.

%\begin{lstlisting}[language=Python, caption=Python example]
\begin{lstlisting}[language=Python]
# -*- coding: utf-8 -*-

import time

INF = 100001
inversions = 0

def MergeAndCount(A, p, q, r):
    global INF
    global inversions
    L = A[p:q+1]
    L.append(INF)	#add a sentinel card
    R = A[q+1:r+1]
    R.append(INF)	#add a sentinel card
    i = 0
    j = 0
    for k in range(p, r + 1):
        if L[i] < R[j]:
            A[k] = L[i]
            i = i + 1
        else:
            A[k] = R[j]
            j = j + 1
            if L[i] != INF:
                inversions = inversions + len(L) - i - 1

def SortAndCount(A, p, r):
    if p < r:
        q = int((p + r) / 2)
        SortAndCount(A, p, q)
        SortAndCount(A, q + 1, r)
        MergeAndCount(A, p, q, r)

if __name__ == "__main__":
    Q5 = open('Q5.txt', encoding = 'utf-8')
    data = [ int(x) for x in Q5 ]
    Q5.close()
    start = time.clock()
    SortAndCount(data, 0, len(data) -1 )
    end = time.clock()
    print("number of inversions:%d\ntime:%f s"%(inversions,end-start))

\end{lstlisting}

The number of inversions in Q5.txt is 2500572073, running time is 1.658 s

\subsection*{\textnormal{7.2\quad Quick-Sort version}}

Yes! Quick-Sort can also count inversions. Typical Quick-Sort is unstable, so it can't count inversions, once we make it stable, it can.
\begin{lstlisting}[language=Python]
# -*- coding: utf-8 -*-

import time

inversions = 0

def Partition(A, p, r):
    global inversions
    tmp = [0] * (r-p+1)
    pivot = A[p]
    k = 0
    for i in range(p+1, r+1):
        if A[i] < pivot:	#less than pivot
            tmp[k] = A[i]
            inversions = inversions + i - k - p
            k = k + 1
    tmp[k] = pivot
    ans = k + p
    k = k + 1
    for i in range(p+1, r+1):
        if A[i] > pivot:	#greater than pivot
            tmp[k] = A[i]
            k = k + 1
    k = 0
    for i in range(p, r+1):	#copy back
        A[i] = tmp[k]
        k = k + 1
    return ans

def QuickSortAndCount(A, p, r):
    if p < r:
        q = Partition(A, p, r)
        QuickSortAndCount(A, p, q-1)
        QuickSortAndCount(A, q + 1, r)

if __name__ == "__main__":
    Q5 = open('Q5.txt', encoding = 'utf-8')
    data = [ int(x) for x in Q5 ]
    Q5.close()
    start = time.clock()
    QuickSortAndCount(data, 0, len(data) -1 )
    end = time.clock()
    print("number of inversions:%d\ntime:%f s"%(inversions,end-start))

\end{lstlisting}

The number of inversions in Q5.txt is 2500572073, running time is 2.266 s. it is slower than Merge-Sort version, although the complexity is still $O(nlgn)$, it has to scan the array 3 times in Partition step, and it isn't a in-place sort.

\section*{8\quad Divide and Conquer}
Here is my implementation of Find-Closest-Pair in Python3.
\begin{lstlisting}[language=Python]
# -*- coding: utf-8 -*-
"""
Created on Tue Oct  6 16:09:40 2015

@author: czl
"""
import copy
import math

INF = 10000000 # max of the **square** of the distance

class Point:
	x = 0
	y = 0
	def __init__(self, x, y):
		self.x = x
		self.y = y

# calculate the square of the distance of point i and point j
def GetDistanceSquare(i, j):
	return (i.x - j.x) * (i.x - j.x) + (i.y - j.y) * (i.y - j.y)

def FindClosestPair(s, e):
    global INF
    if e - s < 3: # if less than 3 points, just brute force
        local_min1 = [INF,0,0]
        for i in range(s, e):
            for j in range(i + 1, e + 1):
                if GetDistanceSquare(dataX[i], dataX[j]) < local_min1[0]:
                    local_min1[0] = GetDistanceSquare(dataX[i], dataX[j])
                    local_min1[1] = dataX[i]
                    local_min1[2] = dataX[j]
        return local_min1
    else: # else divide and conquer
        m = int((s + e) / 2)
        l = FindClosestPair(s, m)
        r = FindClosestPair(m + 1, e)
        local_min2 = []
        if l[0] < r[0]:
            local_min2 = copy.deepcopy(l)
        else:
            local_min2 = copy.deepcopy(r)

        Y = []
        median = dataX[m]

        # collect points within the 2local_min2[0] strip
        # already sorted by y
        for i in dataY:
            if i.x >= median.x - local_min2[0] and i.x <= median.x + local_min2[0]:
                Y.append(i)
        for i in range(0, len(Y)):
            for j in range(i + 1, i + 7): # only calculate next 7 points
                if j >= len(Y):
                    break
                if GetDistanceSquare(Y[i], Y[j]) < local_min2[0]:
                    local_min2[0] = GetDistanceSquare(Y[i], Y[j])
                    local_min2[1] = Y[i]
                    local_min2[2] = Y[j]
        return local_min2

if __name__ == "__main__":
    Q8 = open('Q8.txt', encoding = 'utf-8')
    dataX = []
    for line in Q8:
        v = line.split()
        dataX.append(Point(int(v[0]), int(v[1])))
    Q8.close()
    dataY = copy.deepcopy(dataX)
    dataX.sort(key = lambda p: p.x) # pre-sorted by x
    dataY.sort(key = lambda p: p.y) # pre-sorted by y
    ans = FindClosestPair(0, len(dataX) - 1)
    print('The Closest Pair is (%d,%d)--(%d,%d)\nThe distance is %f'%(ans[1].x, ans[1].y, ans[2].x, ans[2].y, math.sqrt(ans[0])))
\end{lstlisting}

Example input(Q8.txt, each line has 2 numbers indicate a point (x,y)):

43 67

82 35

81 37

70 98

71 94

24 61

21 34

5 2

67 29

42 76
\\

Example output:

The Closest Pair is (81,37)--(82,35)

The distance is 2.236068
\\

Let $T(n)$ be the running time of each recursive step and $T'(n)$ be the running time of the entire algorithm. At the beginning, we sort the data by x and by y, so $T'(n)=T(n)+O(nlgn)$, and
\[
T(n)=
\begin{cases}
2T(n/2)+O(n) & \text{if $n>3$}\\
O(1) & \text{if $n\leq 3$}
\end{cases}
\]
Thus, $T(n)=O(nlgn)$ and $T'(n)=O(nlgn)$.
\end{document} 