% !Mode:: "TeX:UTF-8"
\documentclass[a4paper,12pt]{article}

%插入超链接,并且去除超链接的颜色下划线等特性
\usepackage[hidelinks]{hyperref}

%该宏包可以添加伪代码CLRS《算法导论》**第三版**模式
%clrscode3e.sty放在项目根目录下即可
\usepackage{clrscode3e}

%插入数学公式
\usepackage{amsmath}

%中文包
\usepackage{xeCJK}
\setCJKmainfont{SimSun}

%设置页边距
\usepackage[top=1in,bottom=0.8in,left=1.1in,right=1in]{geometry}

\usepackage{xcolor}

%为了插入代码
\usepackage{listings}

\definecolor{codegreen}{rgb}{0,0.6,0}
\definecolor{codegray}{rgb}{0.5,0.5,0.5}
\definecolor{codepurple}{rgb}{0.58,0,0.82}
\definecolor{backcolour}{rgb}{0.95,0.95,0.92}

\lstdefinestyle{mystyle}{
    backgroundcolor=\color{backcolour},
    commentstyle=\itshape\color{codegreen},
    keywordstyle=\color{magenta},
    numberstyle=\tiny\color{codegray},
    stringstyle=\color{codepurple},
    basicstyle=\footnotesize,
    breakatwhitespace=false,
    breaklines=true,
    captionpos=b,
    keepspaces=true,
    numbers=left,
    numbersep=5pt,
    showspaces=false,
    showstringspaces=false,
    showtabs=false,
    tabsize=4
}

\lstdefinestyle{default}{
    commentstyle=\itshape,
    numberstyle=\tiny,
    basicstyle=\footnotesize,
    breakatwhitespace=false,
    breaklines=true,
    captionpos=b,
    keepspaces=true,
    numbers=left,
    numbersep=5pt,
    showspaces=false,
    showstringspaces=false,
    showtabs=false,
    tabsize=4
}

\lstset{style=default}

%首段缩进
\usepackage{indentfirst}


\title{Assignment 2 Supplement\\Algorithm Design and Analysis}

\author{\href{http://bitjoy.net}{bitjoy.net}}
\date{January 21, 2016}

\begin{document}

\maketitle

\section*{3\quad Partition}

设$dp[i][j]$表示$s[i,...,j]$是否为回文串,$cut[j]$表示$s[1,...,j]$需要多少刀才能切成回文串,则我们需要求$cut[n]$。

对于$cut[j]$,切第一刀的时候,我们尝试前面$j$个切割位点,比如我们尝试切第$i$点,则$s[1,...,j]$被分成了$s[1,...,i-1]$和$s[i,...,j]$,如果$s[i,...,j]$为回文串,则$s[i,...,j]$不需要再切了,又因为$cut[i-1]$我们之前已经算过了,所以在$i$处切一刀,有$cut[j]=cut[i-1]+1$。尝试所有的切割位点$i$,找最小的$cut[j]$。

如果发现$i=1$的时候$s[i,...,j]$为回文串,则$s[1,...,j]$不需要再切了,所以$cut[j]=0$。

$DP$转移公式如下:
\[
cut[j]=\min_{i\in \{i~|~s[i,...,j]~is~palindromic\}}\{cut[j],cut[i-1]+1\}
\]

因为当$i==j$时,只有一个字母,$s[j]$一定为回文,所以至少有一个$i$可满足上式。

\begin{codebox}
\Procname{$\proc{PARTITION}(s)$}
\li n = length(s)
\li dp[n][n] = false
\li \For i = 1 \To n
    \Do
\li dp[i][i] = true
    \End
\li \For j = 1 \To n
    \Do
\li cut[j] = j // set maximum \# of cut
\li \For i = 1 \To j
    \Do
\li \If (s[i] == s[j]) \&\& (j - i <= 1 || dp[i + 1][j - 1])
    \Then
\li dp[i][j] = true
\li \If i > 1
    \Then
\li cut[j] = min(cut[j],cut[i - 1] + 1)
\li \ElseIf i == 1
    \Then
\li cut[j] = 0
    \End
    \End
    \End
    \End
\li \Return cut[n]
\end{codebox}

时间复杂度为$O(n^2)$。

\section*{4\quad Subsequence Counting}

设$dp[i][j]$表示由$S[1,...,i]$变到$T[1,...,j]$有多少种subsequence方法。此时我们有两种选择,如果$S[i]\neq T[j]$,则等价于由$S[1,...,i-1]$变到$T[1,...,j]$,所以$dp[i][j]=dp[i-1][j]$;如果$S[i]=T[j]$,则还可以由$S[1,...,i-1]$变到$T[1,...,j-1]$,所以$dp[i][j]+=dp[i-1][j-1]$。

DP转移公式如下:

\[
dp[i][j]=
\begin{cases}
dp[i-1][j] & \text{if $S[i]\neq T[j]$}\\
dp[i-1][j]+dp[i-1][j-1] & \text{if $S[i]=T[j]$}\\
\end{cases}
\]

初始值$dp[i][0]=1$,因为由任意字符串转换为空字符串只有一种方法,就是把$S$中的字符全删了;$dp[0][j]=0$,因为空字符串的subsequence不可能是是某个非空字符串。

\begin{codebox}
\Procname{$\proc{SUBSEQ-COUNTING}(S,T)$}
\li n = length(S); m = length(T)
\li dp[n][m] = 0
\li \For i = 0 \To n
    \Do
\li dp[i][0]=1
    \End
\li \For i = 1 \To n
    \Do
\li \For j = 1 \To m
    \Do
\li dp[i][j] = dp[i-1][j]
\li \If S[i] == T[j]
    \Then
\li dp[i][j] += dp[i-1][j-1]
    \End
    \End
    \End
\li \Return dp[n][m]
\end{codebox}

时间复杂度为$O(nm)$。
\end{document} 